\documentclass[../modello-progetto.tex]{subfiles}
\begin{document}

In questa sezione verrà presentato il modello del problema di programmazione lineare che verrà successivamente implementato tramite il linguaggio di programmazione matematica AMPL.

\subsection{Insiemi}
\label{sec:insiemi}
Si definiscono i seguenti insiemi:
\begin{itemize}
	\item C è l'insieme dei cuochi che si candidano per partecipare alla cena di gala.
	\item P è l'insieme delle portate che verranno preparate nel corso della serata.
	\item A è l'insieme del personale aggiuntivo (quindi cuochi esclusi) che è necessario assumere.
\end{itemize}
Nel caso specifico del problema presentato abbiamo che:
\begin{itemize}
	\item C: \{A, B, C, D, E, F, G, H, I, L, M, N\} è l'insieme dei cuochi.
	\item P: \{\textit{antipasto}, \textit{primo piatto}, \textit{secondo primo piatto}, \textit{secondo}, \textit{dessert} \} è l'insieme delle portate nell'ordine presentato nella tabella.
	\item A: \{ \textit{aiuto cuoco}, \textit{cameriere} \} è l'insieme delle persone che il proprietario del ristorante deve assumere eccetto i cuochi.
\end{itemize}

\subsection{Variabili}
\label{sub:variabili}

Si definiscono le variabili che verranno utilizzate:
\begin{itemize}
	\item $x_{ij} =\begin{cases} 1, \mbox{ se il cuoco }  i \in C \mbox{ cucina la pietanza } j \in P \\ 0, \mbox{ altrimenti} \end{cases} $
	\item $y_i =\begin{cases} 1, \mbox{ se il cuoco }  i \in C \mbox{ prepara almeno una portata }\\ 0, \mbox{ altrimenti} \end{cases}$
	\item $w_i =\begin{cases} 1, \mbox{ se il cuoco }  i \in C \mbox{ prepara una portata aggiuntiva}\\ 0, \mbox{ altrimenti} \end{cases}$
	\item $u_i =\begin{cases} 1, \mbox{ se il cuoco }  i \in C \mbox{ prepara un dessert }\\ 0, \mbox{ altrimenti} \end{cases}$
	\item $z_i : \mbox{quantità di personale } i \in A \mbox{ assunto}$.
\end{itemize}

\subsection{Parametri}
\label{sub:parametri}

Si definiscono i parametri che verranno utilizzati, i cui valori varieranno in base al file .dat che si sceglierà di usare:
\begin{itemize}
	\item $necessita_j \in \mathbb{Z}^+: \mbox{numero di cuochi necessari per la portata j} \in P$.
	\item $voto_{ij} \in \mathbb{Z}^+: \mbox{voto ottenuto dal cuoco } i \mbox{ nella portata } j $.
	\item $compenso_i \in \mathbb{R}^+: \mbox{ compenso (in euro) promesso al cuoco } i \in C \mbox{ per ogni portata } j \in P \mbox{ che realizza}$.
	\item $stipendio_k \in \mathbb{R}^+: \mbox{ stipendio(in euro) promesso al personale } k$.
	\item $ maxPortate \in \mathbb{Z}^+: \mbox{numero di portate massimo che ogni cuoco } i \in C \mbox{ può realizzare}$.
	\item $ surplus \in \mathbb{Z}^+: \mbox{surplus previsto per l'eventuale cuoco } i \in C \mbox{ che realizzerà la portata aggiuntiva}$.
	\item $ portateTotali \in \mathbb{Z}^+: \mbox{numero di portate totali}$.
	\item $ budgetTotale \in \mathbb{R}^+: \mbox{budget totale a disposizione del ristorante}$.
	\item $ M \in \mathbb{R}^+: \mbox{ costante di big M sufficientemente grande, utile ad attivare le variabili binarie} \\ \mbox{Il suo valore verrà posto pari a } 10^7$.
\end{itemize}

\subsection{Modello}
\label{sub:modello}

\textbf{Funzione obiettivo:}
\[
	max \: \: \: \: \frac{\sum_{j \in P}  \frac{\sum_{i \in C}{x_{ij} \: v_{ij}}}{necessita_j}}{portateTotali}
\]
\par È necessario dividere per il numero di portate totali per ottenere la qualità media della cena. \newline

\noindent \textbf{Vincoli:} \\
La portata j richiede un numero di cuochi ben preciso, specificato nell'array  $necessita_j$:

\[
	\sum_{i \in C} x_{ij} \geq necessita_j  \: \: \: \: \forall j \in P
\]
\[
	\sum_{i \in C} x_{ij} \leq necessita_j  \: \: \: \: \forall j \in P
\]

\noindent Ogni cuoco può realizzare un numero limitato di portate pari a maxPortate:

\[
	\sum_{j \in P} x_{ij} \leq  maxPortate + w_i \: \: \: \: \forall i \in C
\]

\noindent Tuttavia, al più un cuoco può preparare una portata aggiuntiva:
\[
	\sum_{ i \in C} w_i \leq 1
\]
Il ristorante ha a disposizione un budget massimo che deve usare per pagare ristoratori, aiuto cuochi e camerieri:
\[
	\sum_{i \in C} \sum_{j \in P} x_{ij} \: compenso_i+ \sum_{k \in A} stipendio_k \: z_k + \sum_{i \in C}surplus \: w_i \leq budgetTotale
\]
Il numero di camerieri assunti deve essere almeno 2/3 del numero di aiuto cuochi :
\[
	z_{cameriere} \geq  \frac{2}{3} \cdot z_{aiuto\:cuochi}
\]
Per comprendere quanti siano i cuochi effettivamente assunti è necessario introdurre una variabile che ci dica se il cuoco i \in C cucina almeno una portata:

\[
	M \: y_i \geq \sum_{j \in P} x_{ij} \: \: \: \forall i \in C
\]
Inoltre, come stabilito nel problema, il numero di addetti del personale da assumere  deve essere almeno il doppio del numero di cuochi che partecipano alla cena:

\[
	\sum_{k \in A} z_k \geq 2 \: \sum_{i \in C} y_i
\]
I cuochi che preparano il dessert possono dedicarsi unicamente a quello. Per tale ragione si introduce una variabile che determina se il cuoco i \in C prepari o meno il dessert:

\[
	u_i \geq x_i,_{dessert} \: \: \forall i \in C  \: \: \: \: (attivazione \; variabile)
\]
Imponiamo ora la condizione che assicura che un cuoco che prepara il dessert non prepari altri piatti:

\[
	portateTotali - 1 \: (1-u_i) \geq \sum_{j \in P \backslash \{dessert\}} x_i,_j \: \: \: \: \forall i \in C
\]
Infine imponiamo il vincolo che stabilisce che la media dei voti dei cuochi che preparano il dessert deve essere almeno pari a 8:


\textbf{Domini delle variabili}:
\begin{center}
	$ x_i,_j  \in \{0,1\} $

	$ y_i \in \{0,1\} $

	$ w_i  \in \{0,1\} $

	$ u_i  \in \{0,1\} $

	$ z_i  \in {Z}^+ $

 \end{center}
\end{document}